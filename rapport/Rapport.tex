\documentclass[a4paper,twoside,12pt]{report}
\usepackage[utf8x]{inputenc}
\usepackage{color}
\usepackage{fancyvrb} % pour mettre les verbatim dans des boites
%\usepackage[dvipdf]{graphics}
\usepackage[pdftex]{graphicx}
\usepackage{fancyhdr}
\usepackage{listings}
\usepackage{float}
\usepackage{listingsutf8}
\usepackage[T1]{fontenc}
\usepackage[french]{babel}
\usepackage{textcomp}
\usepackage[toc,page]{appendix}
%\usepackage{hyperref}
% Quelques r\'eglages particuliers
\oddsidemargin=-0.1in %xx
\evensidemargin=-0.1in
\textwidth=6.1in
\topmargin=-0.5in
\textheight=8.7in
\parskip 0.25in
\lstset{
language=Java,
basicstyle=\normalsize,
upquote=true,
aboveskip={1.5\baselineskip},
columns=fullflexible,
showstringspaces=false,
extendedchars=true,
breaklines=true,
showtabs=false,
showspaces=false,
showstringspaces=false,
identifierstyle=\ttfamily,
keywordstyle=\color[rgb]{1,0,1},
commentstyle=\color[rgb]{0.133,0.545,0.133},
stringstyle=\color[rgb]{0.627,0.126,0.941},
}
\fancyhead[RO,RE]{\includegraphics[width=0.5in]{Logo-univ-orleans.png}}
\fancyhead[CO,CE]{Rapport de projet}
\fancyhead[LO,LE]{Jeu de Taquin}
\fancyfoot[C]{2015}
\fancyfoot[RO, LE] {\thepage}
%\fancyfoot[LO, RE] { ARNOULT S., MEKHILEF W., OUSSAD J., RETY M.}
%
%Quelques macros
\newcommand{\moncode}[1]{\begin{center}
                        \lstinputlisting[inputencoding=utf8/latin1]{#1}
                        \end{center}}

\newcommand{\monimage}[4]{
\par\noindent
\begin{figure}[H] %on ouvre l'environnement figure
\begin{center}
\includegraphics[width=#4\textwidth]{#1} %ou image.png, .jpeg etc.
\caption{#2} %la l\'egende
\label{#3} %l'\'etiquette pour faire r\'ef\'erence \`a cette image
\end{center}
\end{figure} %on ferme l'environnement figure
}

\newcommand{\ml}[0]{\par\noindent}

%\hyphenation{suppl\'eentaires}{sup-pl\'e-men-tai-res}
%opening

\begin{document}

\newpage
\pagestyle{empty}
%
\begin{figure}[H]
\includegraphics[width=0.2\linewidth]{Logo-univ-orleans.png}
% \hfill

\end{figure}
\vspace{2cm}
%
\begin{center}
{\Huge Licence 2 Informatique\\\ \\Rapport du projet IF06}
\par\vspace{1.4cm}

{\Huge\bf \textcolor{red}{\bf Jeu de Taquin}}
\par\vspace{1.6cm}

{\Large       R\'ealis\'e par:}
\par\vspace{1.3cm}
{\large\bf \textcolor{blue}{ARNOULT Simon, MEKHILEF Wissame, OUSSAD Jihad, RETY 
Martin}}
\vfill
\today
\end{center}
\newpage
\pagestyle{fancy}

\begin{abstract}
%
Nous avons le plaisir de vous pr\'esenter notre travail sur ce projet. Durant la 
deuxi\`eme ann\'ee de la l2 Informatique
\`a l'Universit\'e d'Orl\'eans, nous avons travaill\'e en groupe de quatre sur un projet 
de r\'esolution de Taquin.
\end{abstract}
 
\newpage
\tableofcontents
\listoffigures
\newpage

\chapter{Introduction}
Nous allons ensemble, aborder quatre points principaux dans ce rapport:
\begin{itemize}
\item L'\'etude du projet
\item L'organisation de travail au sain du groupe
\item Le d\'eveloppement du code
\item L'analyse des algorithmes.
\end{itemize}
%
\chapter{Etude}
\par
L'\'etude du projet fut une premi\'ere \'etape importante pour se mettre dans une 
bonne dynamique de groupe. D\`es la premi\`ere semaine nous avons r\'ealis\'e une 
version jouable du Taquin, puis nous avons travaill\'e sur cette version durant 
toute la premi\`ere phase d'essais.
\par\noindent
Rapidement il nous a fallu faire des essais sur ce jeu pour qu'il puisse se 
r\'esoudre algorithmiquement. Nous avons chacun travaill\'e de notre cot\'e sur nos 
id\'ees pour pouvoir en tester un plus grand nombre, mais tout en restant en 
contact r\'eguli\`erement pour avancer ensemble.
\par\noindent
Cette mani\`ere de r\'epartir les t\^aches nous a permis de se rendre compte des 
difficult\'es que l'on allait rencontrer plus rapidement.
%
\section{Analyse de faisabilit\'e}
\par
L'analyse de faisabilit\'e fut un moment qui a dur\'e du d\'ebut du projet jusqu'aux 
vacances de f\'evrier, durant cette phase chacun travaillant sur des 
fonctionnalit\'es diff\'erentes, nous avons pu voir ou \'etaient les probl\`emes dans 
notre architecture de d\'epart.
\par\noindent
Ces tests ont \'et\'e tr\`es divers. Nous avons travaill\'e sur les fonctionnalit\'es 
VT100 du terminal et la r\'ecup\'eration des touches tap\'ees par l'utilisateur. Mais 
nous avons aussi cr\'e\'e des algorithmes pour essayer de r\'esoudre le Taquin.
\par\noindent
Tous ces tests nous ont permis au mois de f\'evrier d'avoir une id\'ee claire de 
l'architecture du projet.
\section{Conception UML}
\par
Cette analyse fin f\'evrier a permis d'aboutir au diagramme UML suivant, diagramme 
qui n'a pas beaucoup \'evolu\'e jusqu'\`a la version finale. Pour
faciliter la lecture nous avons divis\'e l'architecture en packages, seulement 3 
des packages sont pr\'esent\'es.
\ml
\monimage{../dev/jeuPackage.png}{Repr\'esentation UML du package jeu}{JeuP}{0.9}
\ml
\monimage{../dev/algoPackage.png}{Repr\'esentation UML du package 
algo}{AlgoP}{0.9}
\ml
\monimage{../dev/automatePackage.png}{Repr\'esentation UML du package 
automate}{AutomateP}{0.3}
\chapter{La gestion de projet}
%
\par
La gestion de projet est \`a la base de tout projet, et encore plus quand il se 
fait avec une \'equipe de quatre personnes.
\section{Le travail de groupe}
%
\par
La gestion de projet fut au coeur de nos pr\'eoccupations avant m\^eme que le projet 
ne d\'emarre, nous avons cherch\'e \`a cr\'eer une \'equipe dynamique. Nous nous sommes 
donc vus r\'eguli\`erement dans les salles de l'Universit\'e. Malgr\'e cel\`a il nous a 
fallu mettre en place des moyens d\'edi\'es pour faciliter le travail et \'eviter une 
d\'egradation de l'entente.
\par\noindent
 D\`es la premi\`ere semaine, nous avons mis en place un d\'ep\^ot git sur le site de 
l'h\'ebergeur GitHub, ce d\'ep\^ot nous a permis d'avoir le r\'eflexe de l'utiliser même 
 si nous avons rencontr\'e quelques probl\`emes, il nous a permis \`a chacun de voir 
l'avancement du projet. Le d\'ep\^ot est disponible \`a cette adresse : https://github.com/wissame95/IF06-Projet.git.
\monimage{DepotGit.png}{D\'epôt sur GitHub}{depot}{0.8}
\par\noindent
 Cependant un tel d\'ep\^ot ne r\'epond pas \`a la question de la communication, chacun 
habitant une ville diff\'erente, nous avons donc communiqu\'e via Skype pour 
rem\'edier \`a ce probl\`eme.

\chapter{Phase de d\'eveloppement}
%
\par
Pour la phase de d\'eveloppement nous avons chacun de notre c\^ot\'e d\'evelopp\'e une 
partie de l'application. Nous nous sommes partag\'es les diff\'erents 
algorithmes, mais aussi les autres fontionnalit\'es comme les tests Junit.
\section{L'architecture}
\par
Nous avons fait le choix de r\'epartir dans diff\'erents packages les classes et 
interfaces. Nous avons eu des choix \`a faire \`a plusieurs niveaux. Par
exemple un choix simple pour la grille de jeu, nous avons choisi une matrice 
d'entier en remplacement d'une ArrayList d'entiers. Mais aussi des choix plus
compliqu\'es, comme la repr\'esentation d'un sommet qui dans notre cas est la classe 
Taquin.
\par\noindent
L'architecture se d\'ecompose en 7 packages : 
\begin{description}
 \item [algo] Contient les interfaces EnsembleATraiter et EnsembleMarque, toutes 
les classes impl\'ementants ces interfaces et une classe Algo.
 \item [jeu] Contient une interface Jeu, la classe Taquin. Mais aussi les 
commandes, les actions, et les positions finales.
 \item [comparateur] Contient deux comparateurs Manhattan et DepthManhattan.
 \item [exceptions] Contient toutes les exceptions que nous avons d\^u cr\'eer.
 \item [automate] Contient les classes relatives \`a l'automate.
 \item [junit] Contient les diff\'erents tests JUnit.
 \item [main] Contient une seule classe, Main. Elle g\`ere la lecture des 
param\`etres et l'execution des m\'ethodes dans les autres classes, c'est
 le lien entre l'utilisateur et l'application.
\end{description}

\section{Les fichiers test}
\par
Rapidement pour v\'erifier le fonctionnement du programme nous avons d\^u cr\'eer des fichiers tests de taquins t\'emoins. Ces fichiers se trouvent dans le dossier taquin
\`a la racine du projet. Nous en avons utilis\'es un petit nombre, avec des tailles diff\'erentes et ``bien m\'elang\'es''.
%
\section{Fonction non impl\'ement\'e}
Nous avons rencontrer des soucis sur l'impl\'ementation de la pile d'action, le probl\`eme intervient au moment de remonter les parents d'un Taquin, il vient
certainement d'un null. Ceci a donc engendrer le fait de ne pas pouvoir tester nos m\'ethodes sur le parcours progressif.
Enfin, nous avons aussi rencontrer des soucis de lecture des Junit et benchmarks dans une classe interne.
\chapter{Analyse et Conclusion}
%
\section{Analyse des Benchmark}
%
\par
Les benchmarks fournis par le module h2, nous ont permis d'am\'eliorer l'efficacit\'e de l'application, mais aussi de montrer le bon fonctionnement des m\'ethodes.
Nous avons donc pour Algo et pour Taquin cr\'e\'es deux classes, une prouvant que le programme tourne correctement et l'autre pour montrer la rapidit\'e d'\'ex\'ecution
avec un probl\`eme de plus en plus grand. Nous avons aussi \'ecrit un test pour l'ensemble incomplet pour observer la r\'esolution et le temps n\'ecessaire \`a celle-ci, en
fonction de la taille de l'ensemble utilis\'e.
%
\subsection{Efficacit\'e des fonctions du Taquin}
%
Ci-dessous vous pouvez voir un r\'ecapitulatif des fonctions co\^uteuses de la classe Taquin.
\monimage{taquinSpeed.png}{Efficacit\'e des fonctions du Taquin}{Eff}{0.6}
%
\subsection{Efficacit\'e des algorithmes}
Les tests suivants ont \'et\'e r\'ealis\'es \`a l'aide de taq1.taq qui est un taquin de taille 3x3. On peut observer les diff\'erentes vitesse de r\'esolution pour le m\^eme Taquin en fonction de
l'algorithme.
\ml
Les algorithmes utilisant un ensemble complet sont plus lent que les algortihmes utilisant un ensemble incomplet. De meme on observe que les tas sont plus rapide que les
files et piles, qui demande le plus de temps de calcule.
\ml
Les algorithmes bas\'ees sur un automate sont les plus efficace, cependant l'efficacit\'e d\'epend de la profondeur de l'apprentissage de l'automate et de la taille du taquin.\'e
\ml
 En effet, le nombre de coup redondant grandit en m\^eme temps que la taille du taquin, il est donc pr\'ef\'erable d'utiliser cette automate sur un taquin de grande taille.
\monimage{algoSpeed.png}{Diagramme des temps d'execution}{tempsExec}{0.6}

\subsection{Etude sur l'ensemble incomplet}
Ci-dessous vous pouvez voir un diagramme montrant le temps n\'ecessaire \`a la r\'esolution d'un taquin avec un ensemble incomplet sur 5 tailles diff\'erentes.
\monimage{ensembleincomplettest2.pdf}{Exemple}{EX}{0.6}

\section{Conclusion}
%
En d\'efinitive, ce projet nous a permis d'acqu\'erir une certaine autonomie, puisque nos simples connaissances ne suffisaient pas \`a impl\'ementer efficacement tous les aspects du projet.
Nous avons \'egalement d\^u faire preuve de rigueur et d'inventivit\'e afin de contourner certaines difficult\'es inh\'erentes \`a la complexit\'e algorithmique du projet.
\chapter{Resources utilis\'ees}
\begin{itemize}
 \item JUNIT
 \item JAVA
 \item Dia
 \item Eclipse
 \item Argouml
 \item \LaTeX (sous kile)
 \item Google Drive
 \item Kate
 \item Git, GitHub
\end{itemize}

\end{document}
